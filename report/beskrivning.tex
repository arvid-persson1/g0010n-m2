\section{Verksamhet}

Tullverket är en förvaltningsmyndighet under finansdepartementet på
uppdrag av staten~\cite{styrning}. Dess främsta uppgift är att se över Sveriges
varuflöde, och därigenom se till att regler och restriktioner följs, samt att
skatter och andra avgifter betalas~\cite{verksamhet}.
%
Detta i syfte att förhindra handel och smuggling i olagliga eller annars
person- eller miljöfarliga produkter~\cite{wikipedia}.

\subsection{Historik}

Sedan åtminstone 1100-talet har Sverige tagit tull på handelsvaror. Under
Axel Oxenstierna på 1600-talet solidiferades Tullverkets roll och år 1636
fick Tullverket sin första myndighetschef i och med stärkningen av tullväsendet
för tillväxten av landets ekonomi~\cite{wikipedia}. Posten som myndighetschef
finns kvar idag, dock ovanpå en mer komplex
organisationsstruktur~\cite{forordning}. Det dröjde däremot till 1988 innan
Tullverket blev en egen myndighet, då under Försvarsdepartementet.

Den så kallade Lilla tullen infördes år 1622 och tog tull på vanliga
konsumtionsvaror som livsmedel, kläder, djur, trä och metall. Tullen var på
då varor fördes in i en stad, något som var vanligt både i Sverige och i resten
av Europa på den tiden. Den så kallade Stora sjötullen infördes senare på
utrikeshandel. Dessa avskaffades på 1810-talet.

Fram till 1860 hade tullmännen granskat och fört journal på passhandlingar vid
in- och utresande. Med det växande resandet som följd av färdmedel såsom
järnväg och ångfartyg avskaffades passtvångat, och när det senare återinfördes
på 1900-talet låg då ansvaret hos Polisen, vilket det gör än idag.

Tullar har många gånger justerats för att försäkra svensk konkurrenskraft.
Exempel på detta är under 1700-talet då import av textiler förbjöds med avsikt
att få igång en textilindustri inom landet, eller under den industriella
revolutionen då det blev allt billigare att importera spannmål från USA
och Ryssland.
%
Kontroller har ofta stärkts i samband med ökad smuggling, däribland
spritsmugglingen under mellankrigstiden och narkotikasmugglingen i modern
tid~\cite{historik}.

\subsection{Uppdrag}

Förordning~(2016:1332)~\cite{forordning} utfärdad av finansdepartementet
tilldelar Tullverket ett antal uppgifter. Som icke-kommersiell, statligt
finansierad myndighet är Tullverkets ``produkt'' värdet av den tjänst det
tillför samhället genom att utföra uppdraget.

\subsection{Tullavgifter}

Tullverket ska fastställa och ta ut tullar, skatter och andra avgifter då
varor från utanför EU förs in i Sverige. Inom EU:s tullunion finns inga
tullavgifter. Avgifterna finansierar den offentliga sektorn~\cite{tull}.

Tull sätts utifrån varans ursprung, klassificering och värde baserat på given
deklaration. Tullverket står även för att granska dessa deklarationer och
genomföra riskbedömningar samt eventuella undersökningar eller fysiska
kontroller~\cite{verksamhet}.

Förutom tull debiterar Tullverket också mervärdesskatt och punktskatter,
och ansvarar för att administrera undantag från tullar och skatter genom
frihandelsavtal, speciella program eller liknande.
%
En bedömning görs av dels legalt varuflöde, dels mörkertal beroende på
illegalt varuflöde i form av införsel av bland annat alkohol och tobak
för att uppskatta skattefel.

\subsection{Kontroll av varuflöde}

Tullverket ska kontrollera varuflöde in och ut ur Sverige för att säkerställa
att bestämmelser om in- och utförsel av varor följs. Det gäller alla varor som
passerar gränsen oavsett transportmedel; väg, järnväg, flyg, sjöfart och
annat~\cite{varuflodet}.
%
Eftersom Tullverket inte har egna fartyg går bevakning till sjöss igenom
samverkan med Kustbevakningen~\cite{kustbevakningen}.
%
Tullverket verkar i hela landet, men är speciellt stort vid orter där gränserna
ser mer trafik, bland dessa Stockholm och Malmö~\cite{krisinformation}.

Det är i allmänhet inte Tullverket som sätter dessa restriktioner, utan arbetet
utförs på uppdrag av andra myndigheter. Information om vilken myndighet som
hanterar vilka områden finns öppen. Till exempel är det Läkemedelsverket under
Näringsdepartementet som behandlar restriktioner kring läkemedel och
narkotikaprekursorer, och Polismyndigheten under Justitiedepartementet för
vapen~\cite{kontroller}.

Verksamheten kan innebära att jobba emot organiserad brottslighet i form av
smuggling av narkotika, vapen, ammunition, vari Tullverket samarbetar med andra
myndigheter, bland annat polisen och kustbevakningen.
%
Tullverket har däremot befogenhet att beslagta dessa illegala varor, samt andra
varor med speciella restriktioner såsom alkohol och läkemedel~\cite{beslag}.
%
Det kan även beröra miljöfarliga varor i form av illegalt avfall och djur-
eller växtarter som kan skada det svenska ekosystemet~\cite{wikipedia}.

År 2018 passerade 7 miljoner motorfordon gränsen enbart genom Öresundsbron,
och 28,5 miljoner flyg genom Swedavias flygplatser, 2,7 miljoner av dessa från
utanför Europa. Under samma år ankom kring 1500 containrar och 75000
kurirpaket varje dag~\cite{granskning}.
%
Arbete pågår med att förenkla och effektivisera processerna kring att
inspektera varor. Målet är att endast godkända varor ska komma över gränsen,
utan att det legala varuflödet störs i onödan. Därför utvecklas nya metoder
för analys och riskbedömning~\cite{vision}.
%
För att söka efter narkotika och vapen används sedan 1967 utöver tekniska
hjälpmedel en kår av specialtränade sökhundar~\cite{sokhundar}.

Kontrollen bidrar både till ett tryggare samhälle och starkare konkurrenskraft
för det svenska näringslivet.

\subsection{Andra uppgifter}

Sedan 2008 har Tullverket befogenhet att utföra utandningsprov vid
tullkontroller för att ingripa mot rattfylleri~\cite{rattfylleri}. Det är en
del av ett samarbete mellan ett antal myndigheter för att öka säkerheten
i trafiken.
%
Vid misstanke kan Tullverket också utföra kroppsvisitationer eller vid behov
och med beslut från en åklagare en kroppsbesiktning~\cite{visitation}.

Tullverket ska erbjuda lättillgänglig information och service till
allmänheten såväl som företag så att dessa kan fatta hållbara beslut på
lång sikt~\cite{forordning}. Tanken är att göra det
``lätt att göra rätt''~\cite{vision}.
%
Statistik över Tullverkets beslagtagande finns tillgänglig på
hemsidan~\cite{statistik}, ordnad efter tid, varutyp och plats. Under första
halvåret av år 2024 beslagtogs exempelvis 4,1 miljoner styck, 5000 kilo och 91
liter av diverse typer av narkotika (olika varor mäts i olika enheter).
