\section{Hållbarhetsutmaningar}

De hållbarhetsutmaningar som Tullverket måste hantera för att nå de globala
målen för en hållbar utveckling samt uppfylla de internationella normer,
riktlinjer och standarder kan delas in i tre kategorier: social hållbarhet,
ekonomisk hållbarhet och ekologisk hållbarhet. Genom att uppfylla dessa tre
kategorier kan en verklig hållbarhet uppnås~\cite{hallbar}.

\subsection{Social hållbarhet}

Tullverkets centrala uppgift är att kontrollera varuflöde in och ut ur Sverige
för att säkerställa att bestämmelser om in- och utförsel av varor följs.
Riskfyllda situationer kan förekomma när arbetare inom tullverket hanterar
smugglade varor, kemikalier eller vid konfliktsituationer med okända
individer~\cite{vapen}. FN:s globala mål nummer 3: ``hälsa och välbefinnande'',
handlar om att säkerställa god hälsa och välbefinnande för alla individer,
inklusive arbetare inom Tullverket. Om detta globala mål ska uppfyllas gäller
det att Tullverket säkerhetsställer att alla deras arbetare har rätt
utbildning, rätt utrustning och tillgång till en bra sjukvård.

Arbetet berör även FN:s globala mål nummer 16: ``fredliga och inkluderande
samhällen'', vilket inkluderar delmålet om att skapa en värld där alla har lika
rätt till rättvisa och skydd under lagen. Människor vid gränskontroller
befinner sig ofta inom sårbara situationer. Detta beror ofta på att det finns
en språkbarriär och eftersom alla inte kan deras rättigheter. En
hållbarhetsutmaning blir att respektera mänskliga rättigheter vid
gränskontroll. Gränskontroller involverar ofta beslut baserat på riskbedömning
av arbetare i Tullverket. Detta kan medföra att etnisk profilering riskerar att
uppkomma, vilket strider mot mänskliga rättigheter~\cite{etik}. Om Tullverket 
vill uppfylla FN:s globala 16 och ta ett steg närmare social hållbarhet gäller
det att de utbildar sin personal inom mänskliga rättigheter, skapar ett system
där klagomål kan göras mot Tullverket och skapar en arbetskultur där dessa
klagomål tas seriöst.

Målet har dessutom delmålet ``väsentligt minska alla former av korruption och
mutor''. Tullverket är särskilt sårbara mot mutor och korruption, därmed är en
hållbarhetsutmaning för Tullverket att bekämpa korruption och förmedla rättvisa
processer~\cite{granskning2008}. För att uppfylla detta delmål krävs det att
Tullverket är tydliga med deras regler och riktlinjer och regelbundet utbildar
sin personal.

\subsection{Ekologisk hållbarhet}

Tullverket jobbar mot FN:s globala hållbarhetsmål~\cite{fnmal} i Agenda 2030.
Tullverket hanterar mycket narkotika, och farliga föremål. De jobbar därför med
att skydda den biologiska mångfalden men även med ansvarsfull kemikalie- och
avfallshantering. De bidrar också till den hållbara utvecklingen genom att
ställa och beakta hållbarhetskrav som bidrar till minskat avfall och gynnar
återvinning och återbruk~\cite{policy2021}.

Företaget jobbar även med att minska sin miljöpåverkan genom att utbilda
medarbetarna inom miljöarbetet och miljöområdet vilket kan kopplas till delmål
FN:s globala delmål nummer 13.3: ``Bekämpa klimatförändringarna''. Delmålet
går ut på att förbättra utbildningen och medvetenheten kring
klimatförändringar. Det finns tydliga miljökrav vid upphandling av varor och
tjänster och de ska ta hänsyn till miljön i projektarbeten~\cite{miljo}. Detta
kan kopplas samman med FN:s globala mål nummer 12: ``Hållbar konsumtion och
produktion'' som innebär bland annat att uppmuntra företag till hållbara
offentliga upphandlingsmetoder.  De ska minska påverkan med resor och
energianvändning. Alltså ska deras resor och transporter göras fossilfria och
bli färre, vilket skulle kunna göras genom att digitalisera mer. 

\subsection{Ekonomisk hållbarhet}

Tullverket är en statlig myndighet som lyder under finansdepartementet. Detta
innebär att tullverket är direkt beroende av den svenska regeringens
resurser~\cite{styrning}. En stark svensk ekonomi medför att Tullverket förses
med de resurserna som de behöver. Ett bra arbete av Tullverket stärker i sin
tur statens intäkter.

En av Tullverkets ekonomiska hållbarhetsutmaningar är att smidigt hantera den
ökade handelsvolymen samtidigt som brottsligheten stoppas. Genom att
effektivisera handelsprocessen kommer större volymer kunna hanteras vilket
innebär att företag kommer ha fler möjligheter att göra affärer. Detta i
samband med att brottsligheten stoppas kommer att medföra att den svenska
ekonomin kommer att gynnas, då lagliga företag kommer att växa och bidra till
statens intäkter~\cite{handeln}. FN:s globala mål 8: ``anständiga arbetsvillkor
och ekonomisk tillväxt'', har delmålet ``hållbar ekonomisk tillväxt''. Genom
att arbeta med dessa hållbarhetsutmaningar kommer detta delmål att uppfyllas. 
