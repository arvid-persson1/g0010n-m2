\section{Hållbarhetsarbete}

\begin{quotation}
    ``Hållbarhetspolicyns syfte är att ge uttryck för Tullverkets vilja att
    bidra till en hållbar samhällsutveckling. Det vill säga att i genomförandet
    av myndighetens uppdrag agera för att värna människors värdighet, att
    skydda ekosystemen och eftersträva en ekonomi som respekterar planetens
    integritet och gränser, nu och i framtiden.''
\end{quotation}

Så börjar Tullverkets hållbarhetspolicy~\cite{policy2021} som trädde i kraft år
2021. Där beskrivs hur Tullverket ser sig själv agera för att bidra till den
hållbara utvecklingen. Tullverket har antagit de utmaningar som FN:s globala
hållbarhetsmål~\cite{fnmal} sätter, tillsammans med sju egna:

\subsection{Policy}

\begin{quotation}
    ``Genom att övervaka och kontrollera varuflödet in och ut ur Sverige bidrar
    Tullverket till att förebygga drogmissbruk, minska dödligt våld i
    samhället, skydda biologisk mångfald och främja en ansvarsfull kemikalie-
    och avfallshantering.''
\end{quotation}

Arbetet innefattar alltså att förhindra smuggling av illegala varor, exempelvis
narkotika, och därför även drogmissbruk. Andra illegala varor kan vara
skjutvapen, sprängämnen, dopningspreparat, alkohol och tobak.

\begin{quotation}
    ``Genom att säkerställa en korrekt uppbörd bidrar Tullverket till att
    konkurrens sker på lika villkor.''
\end{quotation}

Målet refererar till det ansvar som Tullverket har som statlig myndighet att
se till att all import och export sker på lika villkor. Den kan kopplas till
FN:s globala mål nummer 16: ``fredliga och inkluderande samhällen'' då målet
bland annat handlar om att främja rättvisa, byggande av effektiva institutioner
och att säkerställa lika villkor för alla. 

\begin{quotation}
    ``Genom att bedriva en förebyggande och brottsutredande verksamhet
    motverkar Tullverket grov organiserad brottslighet och olagliga finans-
    och vapenflöden.''
\end{quotation}

Tullverket har en brottsutredande verksamhet. Detta kan ses som deras arbete
för att motverka smugglingsbrott och undersökninga de skattebrott som är
kopplade till import och export. På så sätt kan Tullverket förhindra och
begränsa organiserad brottslighet.
%
De håller även en så kallad tullmålsjournal, som består av ett register över
tullmålsärenden. De har även tillgång till en rad andra register såsom ett
beslags- och analysregister, ärenderegister samt gemensamma spanings-,
belastnings- och misstankeregister. Dessa används för att, bland annat,
utreda misstänkt brottslighet.

\begin{quotation}
    ``Genom att i anskaffningar beakta och ställa hållbarhetskrav som gynnar
    återbruk och återvinning samt bidrar till minskade avfallsmängder.''
\end{quotation}

Hållbarhetskrav ställs på den import och export som flödar genom Sverige och
ser till att kraven följs. Detta kan kopplas till FN:s globala mål nummer 13:
``bekämpa klimatförändringarna'' som bland annat handlar om att säkra
livsmedelstrygghet, rent vatten och hållbart nyttjande av naturresurser.

\begin{quotation}
    ``Genom att på ett ansvarsfullt sätt hantera avfall och kemikalier och
    agera i enlighet med riksdagens klimatpolitiska ramverk om inga
    nettoutsläpp av växthusgaser i Sverige 2045.''
\end{quotation}

Det svenska klimatpolitiska ramverk som antogs av riksdagen år 2017 säger att
Sverige inte ska ha några nettoutsläpp av växthusgaser till atmosfären år
2045~\cite{ramverk}.
Regeringen menar att inrikes transporter, förutom flyg, ska ha minskat sina
utsläpp med minst 70 procent år 2030 jämfört med år 2010.

\begin{quotation}
    ``Genom att efterleva inriktningen i den statliga värdegrunden och dess
    principer för en god statsförvaltning.''
\end{quotation}

Målet kan kopplas till FN:s 16:e globala mål: ``fredliga och inkluderande
samhällen'' då det handlar om att främja rättvisa, transparens och effektivitet
i samhällsstyrning. Arbete sker alltså för att säkerställa rättssäkerhet,
ansvarstagande och förtroende för offentliga institutioner.

\begin{quotation}
    ``Genom att motverka korruption i alla dess former och bedriva ett aktivt
    och förebyggande verksamhets- och säkerhetsskydd värnas ett fortsatt högt
    förtroende hos uppdragsgivare, kunder och allmänhet.''
\end{quotation}

Även detta mål berör fredliga och inkluderande samhällen då korruption, mutor
och dålig transparens är antiteser till ett hållbart samhälle.

\subsection{Kritik}

Trots att FN:s globala mål trädde i kraft år 2015 uppdaterade inte Tullverket
sin hållbarhetspolicy förrän år 2021. Den nuvarande hållbarhetspolicyn ersatte
en tidigare policy~\cite{policy2015} som uppdaterades år 2015 som i sin tur
ersatte en policy~\cite{policy2013} från år 2013. Policyn från 2015 sattes i
kraft kort efter att FN:s globala mål hade trätt i kraft och man kan anmärka
att Tullverket inte tog sig an dessa mål förrän 6 år senare. 

Policyerna från år 2013 respektive 2015 kan kritiseras då de endast nämner
Tullverkets arbete från ett miljöperspektiv (notera att dessa tituleras
\emph{Miljöpolicy}, till skillnad från \emph{Hållbarhetspolicy}n från år 2021).
De nämner inte att Tullverket ska sträva för att bidra till en mer hållbar
samhällsutveckling såsom motverkandet av korruption eller efterlevandet av god
statsförvaltning.
%
Man kan dra slutsatsen att Tullverket har sedan de globala målen infördes år
2015 arbetat för att uppdatera deras miljöpolicy så att det nu har blivit en
hållbarhetspolicy med större omfattning som bättre passar de globala målen.

Tullverket framställs i media oftast genom objektiva påståenden och
rapportering av konkreta händelser, däribland beslag av smuggelgods eller
förändringar i ledningen; generaldirektörer som tvingats lämna sina
poster~\cite{dn-gd}. Tullverkets arbete är sällan kritiserat och de senaste
åren har endast några specifika fall av kritik förekommit:

Ett exempel på sådan kritik återfinns i ett
beslut~\cite{dj-jo}~\cite{jo-uttalande} från Justitieombudsmannen~(JO). JO
ifrågasätter Tullverkets beslut att tillåta ett produktionsbolag att filma
deras arbete, vilket resulterade i TV-programmet Gränsbevakarna~Sverige.
Beslutet kritiserades eftersom det kan innebära en risk för att
sekretessbelagda eller integritetskänsliga uppgifter röjs. JO påpekar dessutom
att personer som fastnar i Tullverkets kontroller befinner sig i en utsatt
situation, där de inte har möjlighet att undvika offentlig uppmärksamhet. Detta
väcker frågor om Tullverkets etiska riktlinjer, externa socialt ansvarstagande
och deras ansvar gentemot enskilda individer. 

Intern kritik~\cite{ex-uppror}~\cite{svd-kritik} har också lyfts fram, där
organisationens struktur och ledarskap pekas ut som problematiska.
Generaldirektörens detaljstyrning och bristande förankring av beslut hos
medarbetare har lett till arbetsmiljöproblem och en splittrad organisation.
Dessa faktorer kan påverka arbetsklimatet negativt och visar på brister i den
interna sociala hållbarheten inom myndigheten.

Det finns också extern kritik~\cite{eu-riksdagen}~\cite{eu-portalen}.
Europeiska revisionsrätten (ECA), har flera gånger kritiserat Tullverket för
den låga frekvensen av fysiska kontroller. Kontrollerna är myndighetens
centrala del för att säkerställa efterlevnad av handels- och tullregler samt
för att skydda miljö och hälsa. Vilket i sin tur leder, vid låg frekvens av
kontroller, till ekonomiska konsekvenser, såsom förlorade tullavgifter eller
svårigheter att upptäcka olaglig handel, men också miljömässiga problem om
farligt gods inte identifieras i tid exempelvis.

Det kan vara en prioriteringsfråga för Tullverket, där resurser har fördelats
till andra områden på bekostnad av kontrollerna, vilket i sin tur kan leda till
tvivel om deras strategiska hållbarhet och förmåga att upprätthålla balans
mellan effektivitet, miljöhänsyn och ekonomiska intressen vilket i sin tur
gagnar skattebetalarna.

\subsection{Sammanfattning}

Tullverket borde, utifrån punkterna ovan, utveckla mer transparenta mål för
ekonomisk och miljömässig hållbarhet. Att råda bot på den externa kritiken
som har lyfts fram av ECA skulle stärka Tullverkets trovärdighet samt minska
miljörisker om frekvensen av kontroller ökade mer som efterfrågats. Det sociala
ansvaret borde också stärkas genom att förbättra arbetsmiljön och hantera de
interna problem som finns. Genom att jobba enligt den internationella
standarden ISO~26000~\cite{iso26000} skulle det sociala ansvaret och tilliten
till myndighetens arbete kunna bättras~\cite{boken-soc}. Kommunikationen eller
dialogen med externa intressenter för att få en mer balanserad bild av
hållbarhetsarbetet vore också en förbättring.
