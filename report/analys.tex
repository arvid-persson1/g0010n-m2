\section{Analys}

Som nämnt i avsnitt~\ref{social} bör Tullverket säkerhetsställa sitt arbete
genom att försöka följa FN:s globala mål nummer 3: ``hälsa och välbefinnande''.
Tullverket behöver se till att arbetare har rätt till utbildning, och tillgång
till rätt skyddsutrustning och sjukvård. Detta är inte något som nämns i
Tullverkets hållbarhetsplan. Detta, tillsammans med att det tidigare funnits
kritik över att Tullverket har bristande ledarskap~\cite{uppror}~\cite{fara},
kan leda till att organisationen inte tar detta ansvar på allvar vilket kan
leda till vidare framtida problem.

Något som är viktigt för ett hållbart samhälle som Tullverket tar upp i sin
hållbarhetsplan är bekämpandet av korruption. Att detta är förstärkt i
Tullverkets hållbarhetsplan är en mycket bra sak. Detta betyder att Tullverket
är medvetna och villiga att kämpa mot korruption. 

Tullverket tar upp i sin hållbarhetsplan att en del av främjandet av ett
hållbart samhälle är beskyddandet av den biologiska mångfalden. Som nämnt i
avsnitt~\ref{ekologisk} anses detta som mycket väsentligt och det är väldigt
bra att Tullverket tar upp detta i sin hållbarhetsplan.

I avsntit~\ref{ekonomisk} nämndes att en av Tullverkets hållbarhetsutmaningar
är att på ett effektivt sätt hantera den ökade handelsvolymen samtidigt som
kriminalitet på gränsen stoppas. Tullverket betonar i deras hållbarhetsplan
vikten av att bekämpa brott på gränsen. Däremot har de inte nämnt något om att
effektivisera deras arbete. Detta är viktigt eftersom en ökad handel kommer
gynna Sveriges ekonomi. 

Tullverket har gjort konkreta steg för att minska miljöpåverkan med sina
tjänsteresor genom att öka användningen av digitala möten. De fokuserar dock
enbart på interna aspekten av resor. Utöver det så kan det inte ses några
större insatser för att minska utsläpp från internationella transporter. När
det gäller ekologisk hållbarhet har Tullverket också som mål att minska
pappersanvändning vilket skulle minska det ekologiska avtrycket. Det bör dock
sägas att det saknas mer omfattande analys eller åtgärder mot de potentiellt
negativa miljöeffekterna av ökad digitalisering, såsom energiförbrukning av de
servrar eller datorer som krävs. Tullverket har fokus främst på interna
processförbättringar.

När det gäller de etiska aspekter inom tullhanteringen så är deras huvudansvar
att säkerställa varor som passerar gränserna uppfyller lagens krav. Arbetet är
starkt kopplat till hållbarhetsutmaningen kring etiska handelsfrågor och visar
på ambition för mer rättvis och global handel. Det framgår dock inte hur
omfattande arbetet är eller vilka resurser som tilldelas för att möta
utmaningen.
